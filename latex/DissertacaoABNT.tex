\documentclass[12pt,openright,oneside,a4paper]{abntex2}

\usepackage[brazil]{babel}
\usepackage[utf8]{inputenc}
\usepackage{graphicx}
\usepackage{amsmath}
\usepackage{float}
\usepackage{indentfirst}
\usepackage{setspace}
\usepackage{caption}

\setlength{\parindent}{1.25cm}
\onehalfspacing

\titulo{Modelo de Dissertação com Integração Java, Gnuplot e \LaTeX}
\autor{Seu Nome Completo}
\local{Cidade}
\data{2025}
\instituicao{Universidade Exemplo\\Programa de Pós-Graduação em Engenharia}
\tipotrabalho{Dissertação (Mestrado Profissional)}
\preambulo{Dissertação apresentada como requisito parcial para obtenção do grau de Mestre em Engenharia.}

\begin{document}

\imprimircapa
\imprimirfolhaderosto

\begin{folhadeaprovacao}
    \assinatura{Prof. Dr. Orientador \\ Universidade Exemplo}
    \assinatura{Prof.ª Dr.ª Membro 1 \\ Universidade Exemplo}
    \assinatura{Prof. Dr. Membro 2 \\ Universidade Exemplo}
    \assinaturaconclusao{Cidade, \today}
\end{folhadeaprovacao}

\begin{dedicatoria}
\vspace*{\fill}
\centering
{\normalfont \em A todos que me apoiaram nesta caminhada.}
\vspace*{\fill}
\end{dedicatoria}

\begin{agradecimentos}
Agradeço ao meu orientador, colegas, familiares e amigos que me apoiaram nesta jornada.
\end{agradecimentos}

\begin{resumo}
Esta dissertação apresenta um modelo de fluxo científico automatizado utilizando ferramentas livres como Java, Gnuplot e \LaTeX. Os resultados demonstram a viabilidade de integração de linguagens de programação e ferramentas científicas para elaboração de trabalhos acadêmicos estruturados.
\end{resumo}

\begin{abstract}
This dissertation presents a model of automated scientific workflow using free tools such as Java, Gnuplot, and \LaTeX. The results demonstrate the feasibility of integrating programming languages and scientific tools for the preparation of structured academic works.
\end{abstract}

\tableofcontents
\listoffigures
\cleardoublepage

\section{Introdução}
Nesta dissertação exploramos o uso de ferramentas computacionais livres para automatizar o ciclo de geração de dados,
visualização e documentação científica.
Arquivo Externo.


\chapter{Fundamentação Teórica}
Conceitos de sinais não lineares, funções periódicas e métodos numéricos são fundamentais.


\chapter{Metodologia}
Os dados são gerados por um programa Java e visualizados com Gnuplot. A Figura~\ref{fig:grafico} mostra um exemplo.

\begin{figure}[H]
\centering
\includegraphics[width=0.8\textwidth]{grafico.jpg}
\caption{Gráfico da função seno gerado automaticamente}
\label{fig:grafico}
\end{figure}


\chapter{Resultados}
Os dados obtidos mostram um comportamento esperado da função $\sin(t)$, com boa qualidade visual e numérica.


\pagebreak
\section{Conclusão}
A integração entre Java, Gnuplot e LaTeX oferece uma abordagem poderosa e flexível para automação de relatórios científicos.
%\chapter{Conclusão}
A automação com Gradle, Java, Gnuplot e \LaTeX mostrou-se eficaz e escalável. O modelo pode ser aplicado a diversas áreas científicas.
Segundo \cite{oppenheim}, métodos digitais são essenciais para sinais discretos.



\begin{references}
\noindent
[1] ABNT NBR 14724:2023 — Trabalhos acadêmicos.\\
[2] OPPENHEIM, A. V.; SCHAFER, R. W. Discrete-Time Signal Processing.
\end{references}

\end{document}

