\documentclass[twocolumn,10pt]{article}
\usepackage[utf8]{inputenc}
\usepackage{amsmath, amsthm, amssymb}
\usepackage{graphicx}
\usepackage{caption}
\usepackage{float}
\usepackage{geometry}
\usepackage{setspace}

\geometry{a4paper, margin=2.5cm}
\setstretch{1.2}

\title{\bfseries Análise de Sinais com Geração Automatizada}
\author{Nome do Autor}
\date{}

\begin{document}

\twocolumn[
\maketitle

\begin{abstract}
\noindent
Este trabalho apresenta um fluxo automatizado para análise de sinais senoidais utilizando Java, Gnuplot e LaTeX. O objetivo é ilustrar como integrar diferentes ferramentas para produzir relatórios científicos reprodutíveis.
\vspace{1em}
\end{abstract}
]

\section{Introdução}
Este projeto demonstra a geração de dados e gráficos integrando linguagens e ferramentas com LaTeX.

\section{Geração dos Dados}
Os dados foram gerados com o seguinte algoritmo:

\begin{equation}
y(t) = \sin(t)
\end{equation}

\section{Visualização}
O gráfico da Figura~\ref{fig:grafico} mostra os valores calculados.

\begin{figure}[H]
\centering
\includegraphics[width=0.45\textwidth]{grafico.jpg}
\caption{Sinal senoidal gerado por Java e plotado com Gnuplot}
\label{fig:grafico}
\end{figure}

\section{Conclusão}
A integração entre Java, Gnuplot e LaTeX oferece uma abordagem poderosa e flexível para automação de relatórios científicos.

\end{document}

